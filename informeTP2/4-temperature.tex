\subsection{Temperature}

    El filtro temperatura toma una imagen fuente y genera un
    efecto que simula un mapa de calor. El filtro toma
    los tres componentes del cada pixel, los suma y divide por 3, y califica a eso
    como la temperatura $t$.

    \begin{center}
    $\mathsf{t}_{(i,j)} = \lfloor(\mathsf{src}.r_{(i,j)} + \mathsf{src}.g_{(i,j)} + \mathsf{src}.b_{(i,j)}) / 3\rfloor$
	\end{center}
    
    En función de la temperatura, se determina el color en la imagen destino. 
    Para evitar diferencias con los resultados la cátedra, la 
    temperatura debe truncarse y guardarse en una variable de tipo entero.

    \begin{center}
	\begin{displaymath}
	\mathsf{dst}_{(i,j)}<r,g,b> = \left\{
	\begin{array}{l l}
				<0,0, 128 + t \cdot 4> & \text{si }t < 32\\
				<0, (t - 32) \cdot 4, 255> & \text{si }32 \le t < 96\\
				<(t-96) \cdot 4, 255, 255 - (t-96) \cdot 4> & \text{si }96 \le t < 160\\
				<255, 255 - (t - 160) \cdot 4, 0> & \text{si }160 \le t < 224\\
				<255 - (t - 224) \cdot 4, 0 , 0> & \text{si no} \\
	\end{array}
	\right.
	\end{displaymath}
	\end{center}
	
	El pseudocódigo q resuelve el filtro es el siguiente:
	
\begin{algorithm}[H]
  \begin{algorithmic}[1]
		\FORALL{y:=0 \TO  Height($I_{src}$)}

			\FORALL{x:=0 \TO  Width($I_{src}$)}
			  
				\STATE $ pixel \gets I_{src}(x,y)$
				
				\STATE $Nat $ $ t \gets \lfloor(\frac{Red(pixel)+Green(pixel)+Blue(pixel)}{3}\rfloor$
				
				\IF{$t < 32$}
					
					\STATE $I_{dst}(x,y) \gets DevolverPixel(0,0,128+t \cdot 4)$
				
				\ELSIF{$ 32 \leq t < 96$}
					
					\STATE $I_{dst}(x,y) \gets DevolverPixel(0,(t-32) \cdot 4,255)$

				\ELSIF{$ 96 \leq t < 160$}
				
					\STATE $I_{dst}(x,y) \gets DevolverPixel((t-96) \cdot 4,255, 255-(t-96) \cdot 4)$

				\ELSIF{$ 160 \leq t < 224$}
				
					\STATE $I_{dst}(x,y) \gets DevolverPixel(255, 255-(t-160) \cdot 4, 0)$

				\ELSE
				
					\STATE $I_{dst}(x,y) \gets DevolverPixel(255-(t-224) \cdot 4, 0, 0)$

				\ENDIF
			
			\ENDFOR

		 \ENDFOR

  \end{algorithmic}
  \caption{$temperature (I_{src}, I_{dst})$}
  \label{alg:temperature}
\end{algorithm}	
