\section{Conclusiones y trabajo futuro}

    A partir de los experimentos realizados en este trabajo, se pudo llegar a la conclusión de que las ventajas que brinda el paradigma \textbf{SIMD} a la hora de implementar programas que realicen operaciones altamente paralelizables, como el procesamiento de imágenes, son verdaderamente significativas. Esto queda reflejado en la gran brecha de rendimiento que se observa entre las implementaciones realizadas con dicho paradigma y las que utilizan el lenguaje de programación C.

    Esto siempre debe contraponerse a otro hecho que se hizo presente durante el proceso de implementación: realizar un programa en lenguaje ensamblador resulta, por lo general, considerablemente más difícil que hacerlo en un lenguaje de más alto nivel. El código resultante es menos legible, es más sencillo cometer errores y el proceso de \emph{debugging} se vuelve considerablemente más arduo. Por eso es importante analizar de antemano las características del contexto particular de aplicación, para poder decidir si este esfuerzo adicional realmente vale la pena.

    Ahondando en los detalles más técnicos de la implementación, se intentó la realización de una optimización manual dentro del código ensamblador, sin obtener resultados destacables. La razón de esto es que estructura del código dificultaba ampliamente la realización de dicha modificación. Realizar la optimización de manera efectiva habría implicado modificar gran parte del código ya armado, con un costo casi equivalente al de empezarlo de nuevo con un enfoque distinto; esto es consecuencia de la ya mencionada dificultad que presenta mantener el código programado en este lenguaje.
