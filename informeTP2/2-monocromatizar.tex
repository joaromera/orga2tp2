\subsection{Monocromatizar}

%En nuestro este caso se implementar ́a la funcion de conversion a escala de grises de la siguiente manera:
En este caso se implementará la conversión a escala de grises. Esto es tomando el maximo entre la componente R, G, B y des esa manera asignarselo a cada componente de pixel.

\begin{center}
  $I_{out}(p) = max(R, G, B)$
\end{center}

Para mas detalle brindamos el pseudocódigo aqui. 

\begin{algorithm}[H]
  \begin{algorithmic}[1]
		\FORALL{y:=0 \TO  Height($I_{src}$)}
		 %\FOR x:=0 \TO  Width($I_{src}$)\STEP 1 \DO
			\FORALL{x:=0 \TO  Width($I_{src}$)}
			  
			  \STATE $pixel \gets I_{src}(x,y)$
			  
			  \STATE $Nat$ $ r \gets Red(pixel) $
			  
			  \STATE $Nat$ $g \gets Green(pixel)$
			  
			  \STATE $Nat$ $ b \gets Blue(pixel)$
			  
			  \STATE $Nat$ $max \gets Maximo(r, Maximo(g, b))$
			  
			  \STATE $pixel \gets DevolverPixel(r,g,b)$
			
			  \STATE $I_{dst}(x,y) \gets pixel$
			  
			\ENDFOR

		 \ENDFOR

  \end{algorithmic}
  \caption{$monocromatizar (I_{src}, I_{dst})$}
  \label{alg:monocromatizar}
\end{algorithm}


\subsubsection*{Monocromatizar ASM}

En este filtro procesaremos de a 4 píxeles en cada iteración. %usando las instrucción \textbf{SSE} de la arquitectura. 
Y por cada iteración se realizaran se realizaran los calculos de los del los máximos de cada píxel.

\begin{itemize}

	\item En los registro \textbf{xmm0,xnm4, xmm5} tenemos la copia de los cuatro pixeles q levantamos de memoria.
			Y en \textbf{xnm1, xmm2, xmm3} las mascaras q utilizamos para los Shuffles que utilizaremos para permutar las componente.

		\begin{center}
		   \begin{tabular}{| c | c | c | c || c | c | c | c || c | c | c | c || c | c | c | c |}
			 \hline
			 a & b & g & r & a & b & g & r & a & b & g & r & a & b & g & r \\ \hline

		   \end{tabular}
		   \\ \textbf{xmm0, xmm4, xmm5}
		\end{center}
		 
		\begin{center}
		   \begin{tabular}{| c | c | c | c || c | c | c | c || c | c | c | c || c | c | c | c |}
			 \hline
			 15 & 13 & 13 & 13 & 11 & 9 & 9 & 9 & 7 & 5 & 5 & 5 & 3 & 1 & 1 & 1 \\ \hline
		   \end{tabular}
		   \\  \textbf{Mascara en xmm1 (mask1)}
		\end{center}

		\begin{center}
		   \begin{tabular}{| c | c | c | c || c | c | c | c || c | c | c | c || c | c | c | c |}
			 \hline
			 15 & 14 & 14 & 14 & 11 & 10 & 10 & 10 & 7 & 6 & 6 & 6 & 3 & 2 & 2 & 2 \\ \hline
		   \end{tabular}
		   \\ \textbf{Mascara en xmm2 (mask2)}
		\end{center}

		\begin{center}
		   \begin{tabular}{| c | c | c | c || c | c | c | c || c | c | c | c || c | c | c | c |}
			 \hline
			 15 & 12 & 12 & 12 & 11 & 8 & 8 & 8 & 7 & 4 & 4 & 4 & 3 & 0 & 0 & 0 \\ \hline
		   \end{tabular}
		   \\ \textbf{Mascara en xmm3(mask3)}
		\end{center}

	\item Realizamos el \textbf{shuffle xmm4, xmm1}q nos coloca componente \textbf{g} en las posiciones que podemos observar en el registro xmm4. Y luego calculamos el máximo con la instrucci'on \textbf{pmaxub}. 


		\begin{center}
		   \begin{tabular}{| c | c | c | c || c | c | c | c || c | c | c | c || c | c | c | c |}
			 \hline
			 a & g & g & g & a & g & g & g & a & g & g & g & a & g & g & g \\ \hline
		   \end{tabular}
		   \\ \textbf{xmm4 $\gets$ pshufb xmm4, xmm1}
		\end{center}


		\begin{center}
		   \begin{tabular}{| c | c | c | c || c | c | c | c || c | c | c | c || c | c | c | c |}
			 \hline
			 a & . & . & max(g,r) & a & . & . & max(g,r) & a & . & . & max(g,r) & a & . & . & max(g,r)  \\ \hline
		   \end{tabular}
		   \\ \textbf{xmm4 $ \gets $ pmaxub xmm4, xmm0}
		\end{center}

	\item Realizamos el mismo procedimiento en este paso. Y nos quedaría el valor del Máximo(ver imagen). 
		\begin{center}
		   \begin{tabular}{| c | c | c | c || c | c | c | c || c | c | c | c || c | c | c | c |}
			 \hline
			 a & b & b & b & a & b & b & b & a & b & b & b & a & b & b & b \\ \hline
		   \end{tabular}
		   \\ \textbf{xmm5 $\gets$ pshufb xmm5, xmm2}
		\end{center}

		\begin{center}
		   \begin{tabular}{| c | c | c | c || c | c | c | c || c | c | c | c || c | c | c | c |}
			 \hline
			 a & . & . & max(g,r,b) & a & . & . & max(g,r,b) & a & . & . & max(g,r,b) & a & . & . & max(g,r,b)  \\ \hline
		   \end{tabular}
		   \\ \textbf{xmm4 $\gets$ pmaxub xmm4, xmm5}
		\end{center}


	\item Sólo resta copiar los máximos. Para esto utilizaremos el Shuffle junto con las mascara contenida en xmm3.

		\begin{center}
		   \begin{tabular}{| c | c | c | c || c | c | c | c || c | c | c | c || c | c | c | c |}
			 \hline
			 a & max & max & max & a & max & max & max & a & max & max & max & a & max & max & max  \\ \hline
		   \end{tabular}
		   \\ \textbf{xmm5 $\gets$ pshufb xmm4, xmm3}
		\end{center}

\end{itemize}

Por ultimo resta copiar esto a memoria. 
Como dato abajo brindamos el código ASM.

\begin{codesnippet}
\begin{verbatim}
section .data
	mask1: db 1, 1, 1, 3, 5, 5, 5, 7, 9, 9, 9, 11, 13, 13, 13, 15
	mask2: db 2, 2, 2, 3, 6, 6, 6, 7, 10, 10, 10, 11, 14, 14, 14, 15
	mask3: db 0, 0, 0, 3, 4, 4, 4, 7, 8, 8, 8, 11, 12, 12, 12, 15

section .text

monocromatizar_inf_asm:
.........
		movdqu xmm0, [rdi]; xmm0=|a b g r|a b g r|a b g r|a b g r|
		movdqu xmm4, xmm0; xmm4= xmm0
		movdqu xmm5, xmm0; xmm5= xmm0

		pshufb xmm4, xmm1; xmm4=|a g g g|a g g g|a g g g|a g g g|
		pmaxub xmm4, xmm0; xmm4=|a . . max(g,r)|a . . max(g,r)|a . . max(g,r)|a . . max(g,r)|;primer maximo
 
		pshufb xmm5, xmm2; xmm5=|a b b b|a b b b|a b b b|a b b b|			
		pmaxub xmm4, xmm5; xmm4=|a . . max(g,r,b)|a . . max(g,r,b)|a . . max(g,r,b)|;maximo de maximos
						 ; xmm4=|a . . max|a . . max|a . . max|a . . max|	
		pshufb xmm4, xmm3; xmm4=|a max max max|a max max max|a max max max|a max max max|

		movdqu [rsi], xmm4; [rsi]= xmm4
......
\end{verbatim}
\end{codesnippet}
