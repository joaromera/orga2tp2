\subsection{Blit}

% Esta operaci\'on recibe m\'as par\'ametros que las dem\'as; adem\'as de tener las im\'agenes fuente y destino (de tama\~no w x h), 
% se recibe una imagen adicional \textbf{blit} tambi\'en con su respectivo tama\~no (bw x bh), que se utilizar\'a a modo de m\'ascara.
% Se cumple que $bw \leq w \wedge bh \leq h$.\\

% La aplicaci\'on de este filtro consiste en generar una nueva imagen combinando la original con el \textbf{blit}.
% Un color de la imagen \textbf{blit} es considerado como transparente,
% para que en la combinaci\'on algunos pixeles del \textbf{blit} queden por encima de la imagen original y otros no se vean.

% Para este trabajo pr\'actico, la imagen \textbf{blit} ser\'a una imagen de Per\'on.
% Decimos que una im\'agen ha sido ``Peronizada'' cuando en su extremo superior derecho podemos ver a Per\'on agitando su brazo.
% % La aplicaci\'on de este filtro consiste en generar una nueva imagen combinando la original con la de Per\'on (blit) para as\'i 
% % obtener una versi\'on ``Peronizada'' de la im\'agen fuente.

% \vskip 8pt
% Para cada p\'ixel $p$ en la imagen de Per\'on (blit):\\
% $dst(p) = \begin{cases}
%     src(p) & \mathrm{si \;} blit(p) \mathrm{\; es \; de \; color \; magenta, \; es \; decir \; sus \; colores \; son \;} (255, 0, 255)\\
%     blit(p) & \mathrm{si \; no}
% \end{cases}$ \\

% Es decir que si el p\'ixel en la imagen de Per\'on es magenta, entonces el nuevo p\'ixel tendr\'a los colores del p\'ixel correspondiente 
% en la imagen original; y si no los colores del p\'ixel en la imagen de Per\'on. \\

% Las columnas y filas que no se puedan procesar de esta manera, es decir, 
% las primeras $h - bh$ filas; y de las filas restantes, las primeras $w - bw$ columnas; 
% quedar\'an igual que como estaban originalmente. \\



% \begin{algorithm}[H]
%   \begin{algorithmic}[1]
% 		\FORALL{y:=0 \TO  Height($I_{src}$)}
		
% 			\FORALL{x:=0 \TO  Width($I_{src}$)}
			  
% 			  \STATE $pixel \gets I_{src}(x,y)$
			  
% 			  \STATE $I_{dst}(x,y) \gets pixel$
			  
% 			\ENDFOR

% 		 \ENDFOR
		 
% 		\STATE $Nat$ $ bh \gets $ Height($I_{src}$) $-$ Height($I_{blit}$)
			  
% 		\STATE $Nat$ $ bw \gets $ Width($I_{src}$) $-$ Width($I_{blit}$)

% 		 \FORALL{y:=0 \TO  Height($I_{blit}$)}

% 			\FORALL{x:=0 \TO  Width($I_{blit}$)}
			  
% 			  \STATE $pixel \gets I_{blit}(x,y)$
			  
% 			  \STATE $Nat$ $ r \gets Red(pixel) $
			  
% 			  \STATE $Nat$ $g \gets Green(pixel)$
			  
% 			  \STATE $Nat$ $ b \gets Blue(pixel)$


% 			  \IF{$\neg ( r = 255$ \AND $g = 0$ \AND $b=255)$}
% 				\STATE $I_{dst}(x+bw,y+bh) = I_{blit}(x+bw,y+bh)$ 
% 			  \ENDIF
			  
% 			\ENDFOR

% 		 \ENDFOR


%   \end{algorithmic}
%   \caption{$blit (I_{src}, I_{dst}, I_{blit})$}
%   \label{alg:blit}
% \end{algorithm}

\subsubsection*{Blit ASM}


En este Filtro Blit procesamos de a 4 pixeles, Tenemos dos ciclos en assembler, uno que recorre las columnas de la imagen y otra q recorre las fila.
Cuando itera la fila es en ese momento q toman los 4 pixeles de $I_{src}$ y lo copiamos en $I_{}$.
Hay caso especial cuando nos encontramos en la fila "Height($I_{src}$) $-$ Height($I_{blit}$)" y columna "Width($I_{src}$) $-$ Width($I_{blit}$)", es en ese caso que tenemos o no aplicar el blit segun la formula explicada arriba.
Abajo detallaremos esa parte importante del código ASSEMBLER q desarrolla el blit con un ejemplo.	

\begin{itemize}

	\item En los registro \textbf{xmm0,xnm14} tenemos la copia de los cuatro píxeles q levantamos de memoria, uno es $I_{src}$ y el otro $I_{blit}$ respectivamente.
			Y en \textbf{xnm2, xmm15} las mascaras q utilizamos para la comparación y operaciones lógicas.

		\begin{center}
		   \begin{tabular}{| c | c | c | c || c | c | c | c || c | c | c | c || c | c | c | c |}
			 \hline
			 a & b & g & r & a & b & g & r & a & b & g & r & a & b & g & r \\ \hline
		   \end{tabular}
		   \\ \textbf{xmm0 $\gets$ $I_{src}$ }
		\end{center}

		\begin{center}
		   \begin{tabular}{| c | c | c | c || c | c | c | c || c | c | c | c || c | c | c | c |}
			 \hline
			 A & B & G & R & A & B & G & R & A & B & G & R & A & B & G & R \\ \hline
		   \end{tabular}
		   \\ \textbf{xmm14 $\gets$ $I_{blit}$ }
		\end{center}
		
		 
		\begin{center}
		   \begin{tabular}{| c | c | c | c || c | c | c | c || c | c | c | c || c | c | c | c |}
			 \hline
			 255 & 255 & 0 & 255 & 255 & 255 & 0 & 255 & 255 & 255 & 0 & 255 & 255 & 255 & 0 & 255 \\ \hline
		   \end{tabular}
		   \\  \textbf{Mascara Magenta xmm15 (maskMagenta)}
		\end{center}

		\begin{center}
		   \begin{tabular}{| c | c | c | c || c | c | c | c || c | c | c | c || c | c | c | c |}
			 \hline
			 00h & 00h & 00h & 00h & 00h & 00h & 00h & 00h & 00h & 00h & 00h & 00h & 00h & 00h & 00h & 00h \\ \hline
		   \end{tabular}
		   \\ \textbf{Mascara en xmm2 (maskCeros)}
		\end{center}
		
	\item Mascara para filtrar píxeles valor magenta. A suponemos que hay dos píxeles color magenta en $I_{blit}$ a modo de ejemplo en los píxeles 1 y 3(siguiendo el order de pixel_15,...,pixel_0). A esta mascara la guardamos en xmm12 y xmm15.
	
		\begin{center}
		   \begin{tabular}{| c | c | c | c || c | c | c | c || c | c | c | c || c | c | c | c |}
			 \hline
			 0xFF & 0xFF & 0xFF & 0xFF & 0x00 & 0x00 & 0x00 & 0x00 & 0xFF & 0xFF & 0xFF & 0xFF & 0x00 & 0x00 & 0x00 & 0x00 \\ \hline
		   \end{tabular}
		   \\ \textbf{xmm12,xmm15 $\gets$ pcmpeqd xmm15, xmm14}
		\end{center}

	\item Mascara para filtrar píxeles que No son magenta.

		\begin{center}
		   \begin{tabular}{| c | c | c | c || c | c | c | c || c | c | c | c || c | c | c | c |}
			 \hline
			 0x00 & 0x00 & 0x00 & 0x00 & 0xFF & 0xFF & 0xFF & 0xFF & 0x00 & 0x00 & 0x00 & 0x00 & 0xFF & 0xFF & 0xFF & 0xFF \\ \hline
		   \end{tabular}
		   \\ \textbf{xmm15 $\gets$ pcmpeqd xmm15, xmm13}
		\end{center}

	\item  Me quedo con los valores de $I_{src}$ que tengo q poner en la imagen $I_{dst}$.
		\begin{center}
		   \begin{tabular}{| c | c | c | c || c | c | c | c || c | c | c | c || c | c | c | c |}
			 \hline
			 A & B & G & R & 0x00 & 0x00 & 0x00 & 0x00 & A & B & G & R & 0x00 & 0x00 & 0x00 & 0x00 \\ \hline
		   \end{tabular}
		   \\ \textbf{xmm12 $\gets$ pand xmm12, xmm0}
		\end{center}		

	\item Me quedo con los vamlores de $I_{blit}$ que tengo q poner en la imagen $I_{dst}$.
		\begin{center}
		   \begin{tabular}{| c | c | c | c || c | c | c | c || c | c | c | c || c | c | c | c |}
			 \hline
			 0x00 & 0x00 & 0x00 & 0x00 & a & b & g & r & 0x00 & 0x00 & 0x00 & 0x00 & a & b & g & r \\ \hline
		   \end{tabular}
		   \\ \textbf{xmm15 $\gets$ pand xmm15, xmm14}
		\end{center}		
	
	\item Junto todos los valores en xmm15.
		\begin{center}
		   \begin{tabular}{| c | c | c | c || c | c | c | c || c | c | c | c || c | c | c | c |}
			 \hline
			 A & B & G & R & a & b & g & r & A & B & G & R & a & b & g & r \\ \hline
		   \end{tabular}
		   \\ \textbf{xmm15 $\gets$ por xmm15, xmm12}
		\end{center}		
	\item Por ultimo resta guardar estos 4 pixeles(xmm15) en $I_{dst}$.

\end{itemize}

Para mas detalles dejamos el es-tracto de ASM.

\begin{codesnippet}
\begin{verbatim}
	maskMagenta: db 255	 ,0, 255, 255, 255, 0, 255, 255, 255, 0, 255, 255, 255, 0, 255, 255
	maskCero: db 0, 0, 0, 0, 0, 0, 0, 0, 0, 0, 0, 0, 0, 0, 0, 0 

section .text

blit_asm:
		...
		movdqu xmm0, [rdi]; xmm0=|A B G R|A B G R|A B G R|A B G R|; img	
		movdqu xmm14, [r15]; xmm14=|a b g r|a b g r|a b g r|a b g r|; blit
							   ; filtro los valores color magenta
		pcmpeqd xmm15, xmm14   ; xmm15=|FF FF FF FF|00 00 00 00|FF FF FF FF|00 00 00 00| 
		movdqu xmm12, xmm15	   ; xmm12= xmm15 paso la mascara a xmm12 
							   ;filtro los valores que no son magenta
		pcmpeqd xmm15, xmm13   ; xmm15=|00 00 00 00|FF FF FF FF|00 00 00 00|FF FF FF FF|
							   ;Me quedo con los velores de xmm0 que tengo que poner en la imagen
		pand xmm12, xmm0       ; xmm2=|A B G R|00 00 00 00|A B G R|00 00 00 00|
							   ;Me quedo con los valores de blit que tengo que poner en la imagen
		pand xmm15, xmm14      ; xmm15=|00 00 00 00|a b g r|00 00 00 00|a b g r|
							   ; Junto los dos valores en xmm15
		por xmm15, xmm12 	   ;xmm15=|A B G R|a b g r|A B G R|a b g r|
		movdqu [rsi], xmm15	   ; Copio todo a dst
		movdqu xmm15, [maskMagenta]
		movdqu xmm13, [maskCero]
		.....
\end{verbatim}
\end{codesnippet}
