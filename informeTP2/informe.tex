% ******************************************************** %
%              TEMPLATE DE INFORME ORGA2 v0.1              %
% ******************************************************** %
% ******************************************************** %
%                                                          %
% ALGUNOS PAQUETES REQUERIDOS (EN UBUNTU):                 %
% ========================================
%                                                          %
% texlive-latex-base                                       %
% texlive-latex-recommended                                %
% texlive-fonts-recommended                                %
% texlive-latex-extra?                                     %
% texlive-lang-spanish (en ubuntu 13.10)                   %
% ******************************************************** %


\documentclass[a4paper]{article}
\usepackage[spanish]{babel}
\usepackage[utf8]{inputenc}
\usepackage{charter}   % tipografia
\usepackage{graphicx}
%\usepackage{makeidx}
\usepackage{paralist} %itemize inline


\usepackage{algorithm}  % implementacion ondas en C
% sudo apt-get install texlive-science
\usepackage{algorithmic} % implementacion ondas en C


%\usepackage{float}
\usepackage{amsmath, amsthm, amssymb}
%\usepackage{amsfonts}
%\usepackage{sectsty}
%\usepackage{charter}
%\usepackage{wrapfig}
%\usepackage{listings}
%\lstset{language=C}

% \setcounter{secnumdepth}{2}
\usepackage{underscore}
\usepackage{caratula}
\usepackage{url}


% ********************************************************* %
% ~~~~~~~~              Code snippets             ~~~~~~~~~ %
% ********************************************************* %

\usepackage{color} % para snipets de codigo coloreados
\usepackage{fancybox}  % para el sbox de los snipets de codigo

\definecolor{litegrey}{gray}{0.94}

\newenvironment{codesnippet}{%
	\begin{Sbox}\begin{minipage}{\textwidth}\sffamily\small}%
	{\end{minipage}\end{Sbox}%
		\begin{center}%
		\vspace{-0.4cm}\colorbox{litegrey}{\TheSbox}\end{center}\vspace{0.3cm}}



% ********************************************************* %
% ~~~~~~~~         Formato de las páginas         ~~~~~~~~~ %
% ********************************************************* %

\usepackage{fancyhdr}
\pagestyle{fancy}

%\renewcommand{\chaptermark}[1]{\markboth{#1}{}}
\renewcommand{\sectionmark}[1]{\markright{\thesection\ - #1}}

\fancyhf{}

\fancyhead[LO]{Sección \rightmark} % \thesection\ 
\fancyfoot[LO]{\small{Nombre Apellido, Nombre Apellido, Nombre Apellido}}
\fancyfoot[RO]{\thepage}
\renewcommand{\headrulewidth}{0.5pt}
\renewcommand{\footrulewidth}{0.5pt}
\setlength{\hoffset}{-0.8in}
\setlength{\textwidth}{16cm}
%\setlength{\hoffset}{-1.1cm}
%\setlength{\textwidth}{16cm}
\setlength{\headsep}{0.5cm}
\setlength{\textheight}{25cm}
\setlength{\voffset}{-0.7in}
\setlength{\headwidth}{\textwidth}
\setlength{\headheight}{13.1pt}

\renewcommand{\baselinestretch}{1.1}  % line spacing

% ******************************************************** %


\begin{document}


\thispagestyle{empty}
\materia{Organización del Computador II}
\submateria{Segundo Cuatrimestre de 2014}
\titulo{Trabajo Práctico II}
\subtitulo{subtitulo del trabajo}
\integrante{Nombre}{XXX/XX}{mail}
\integrante{Nombre}{XXX/XX}{mail}

\maketitle
\newpage

\thispagestyle{empty}
\vfill
\begin{abstract}
En el presente trabajo se describe la problemática de ...
\end{abstract}

\thispagestyle{empty}
\vspace{3cm}
\tableofcontents
\newpage


%\normalsize
\newpage

%\section{Objetivos generales}
%El objetivo de este Trabajo Práctico es ...



\begin{figure}
  \begin{center}
	\includegraphics[scale=0.66]{imagenes/logouba.jpg}
	\caption{Descripcion de la figura}
	\label{nombreparareferenciar}
  \end{center}
\end{figure}

%\section{Objetivos Generales}
En este tp utilizaremos la tecnica de el procesamiento de la mayor cantidad de datos posible. Esta tecnica de basa en el modelos de ejecución SIMD(Single instruction Multiple Data).
\subsection{SIMD(Single instruction Multiple Data)}
	Se trata de un modelo de ejecución capaz de computar una sola operación sobre un conjunto de múltiples datos. \newline
	Es muy util para procesar audio, video o imágenes, donde aplican algoritmos repetitivos sobre ese mismo conjunto de datos, 
	por ejemplo: fitros, compresores. 
	

\section{Objetivos Generales}
En este tp utilizaremos la tecnica de el procesamiento de la mayor cantidad de datos posible. Esta tecnica de basa en el modelos de ejecución SIMD(Single instruction Multiple Data).
\subsection{SIMD(Single instruction Multiple Data)}
	Se trata de un modelo de ejecución capaz de computar una sola operación sobre un conjunto de múltiples datos. \newline
	Es muy util para procesar audio, video o imágenes, donde aplican algoritmos repetitivos sobre ese mismo conjunto de datos, 
	por ejemplo: fitros, compresores. 
	


%\section{Introduccion}


Más precisamente, se lleva a cabo la implementación de los siguientes dos filtros:

\begin{itemize}
 \item 
\end{itemize}
  La elaboración del trabajo se dividió en dos etapas. En primer lugar, se implementaron ambos filtros tanto en lenguaje C como en lenguaje ensamblador para la arquitectura x86-64 de Intel. En este último caso, se utilizaron las instrucciones SSE de dicha arquitectura, que aprovechan el ya mencionado modelo SIMD para procesar datos en forma paralela.

  Una vez realizadas estas implementaciones, fueron sometidas a un proceso de comparación para extraer conclusiones acerca de su rendimiento. Con este fin, se experimentó con variaciones tanto en los datos de entrada como en detalles implementativos de los mismos algoritmos. De esta manera, se pudo recopilar datos sobre el comportamiento de cada implementación, y contrastar estos resultados con diversas hipótesis previamente elaboradas.
\section{Introduccion}


Se lleva a cabo la implementación de los siguientes 5 filtros:

\begin{center}
 \begin{tabular}{cccc}
   \includegraphics[width=0.2\textwidth]{imagenes/island.png} &
   \includegraphics[width=0.2\textwidth]{imagenes/island-blit.png} &
   \includegraphics[width=0.2\textwidth]{imagenes/island-monocromatizar.png} \\
   Imagen original & Blit & Monocromatizar \\
   \\
   \includegraphics[width=0.2\textwidth]{imagenes/island-ondas.png} &
   \includegraphics[width=0.2\textwidth]{imagenes/island-temperature.png} &
   \includegraphics[width=0.2\textwidth]{imagenes/island-edge.png} \\
   Ondas & Temperature & Edge \\
 \end{tabular}
\end{center}
  
  La elaboración del trabajo se dividió en dos etapas. En primer lugar, se implementaron ambos filtros tanto en lenguaje C como en lenguaje ensamblador para la arquitectura x86-64 de Intel. En este último caso, se utilizaron las instrucciones SSE de dicha arquitectura, que aprovechan el ya mencionado modelo SIMD para procesar datos en forma paralela.

  Una vez realizadas estas implementaciones, fueron sometidas a un proceso de comparación para extraer conclusiones acerca de su rendimiento. Con este fin, se experimentó con variaciones tanto en los datos de entrada como en detalles implementativos de los mismos algoritmos. De esta manera, se pudo recopilar datos sobre el comportamiento de cada implementación, y contrastar estos resultados con diversas hipótesis previamente elaboradas.
\section{Filtros}
\subsection{Blit}

\subsubsection{Descripción}

Esta operación recibe dos imagenes -original y blit- y genera una nueva combinando ambas. La combinación se realiza considerando un determinado color del blit como transparente para que en el resultado final algunos pixeles del blit queden por encima de la imagen original. En el lugar de aquellos que no son interpretados como transparentes quedan los pixeles del original. La imagen obtenida es una superposición de ambas, teniendo transparencias en el blit siempre que el color del pixel sea igual a (255,0,255).

% \begin{center}
% 	\begin{tabular}{cccc}
% 	  \includegraphics[width=0.2\textwidth]{imagenes/lenaBLIT.jpg} \\
% 	\end{tabular}
%    \end{center}

\begin{multicols}{2}
A pedido de la cátedra la imagen \textbf{blit} ser\'a una foto de Per\'on. La misma será ubicada en la esquina superior derecha de la imagen, además la imagen original deberá tener un tamaño mayor a 89 píxeles de ancho por 128 de alto para poder ubicar el blit. Como resultado tendremos una imagen \textit{Peronizada} cuando se aplique la siguiente función para cada p\'ixel $p$ en la imagen de Per\'on (blit):\\
\begin{center}
	\begin{tabular}{cccc}
	  \includegraphics[width=0.3\textwidth]{imagenes/lenaBLIT.jpg} \\
	\end{tabular}
   \end{center}
\end{multicols}

$dst(p) = \begin{cases}
    src(p) & \mathrm{si \;} blit(p) \mathrm{\; es \; de \; color \; magenta, \; es \; decir \; sus \; colores \; son \;} (255, 0, 255)\\
    blit(p) & \mathrm{si \; no}
\end{cases}$ \\

\subsubsection{Implementación C}

% Es decir que si el p\'ixel en la imagen de Per\'on es magenta, entonces el nuevo p\'ixel tendr\'a los colores del p\'ixel correspondiente 
% en la imagen original; y si no los colores del p\'ixel en la imagen de Per\'on. \\

% Las columnas y filas que no se puedan procesar de esta manera, es decir, 
% las primeras $h - bh$ filas; y de las filas restantes, las primeras $w - bw$ columnas; 
% quedar\'an igual que como estaban originalmente. \\

Nuestra implementación en C consiste en dos etapas: primero copiamos los píxeles la imagen original en la imagen destino, luego copiamos los píxeles del blit siempre que los mismos no tengan el color (255,0,255). Como Perón debe estar arriba a la derecha de la imagen final, restamos el tamaño de su imagen al de la original en las dimensiones correspondientes. A continuación el pseudocódigo describe el algoritmo:

\begin{algorithm}[H]
  \begin{algorithmic}[1]
		\FORALL{y:=0 \TO  Height($I_{src}$)}		
			\FORALL{x:=0 \TO  Width($I_{src}$)}
			  \STATE $pixel \gets I_{src}(x,y)$ 
			  \STATE $I_{dst}(x,y) \gets pixel$
			\ENDFOR
		\ENDFOR
		\STATE $Int$ $ bh \gets $ Height($I_{src}$) $-$ Height($I_{blit}$)
		\STATE $Int$ $ bw \gets $ Width($I_{src}$) $-$ Width($I_{blit}$)
		 \FORALL{y:=0 \TO  Height($I_{blit}$)}
			\FORALL{x:=0 \TO  Width($I_{blit}$)}
			  	\STATE $pixel \gets I_{blit}(x,y)$
				\STATE $Int$ $ r \gets Red(pixel) $
			  	\STATE $Int$ $g \gets Green(pixel)$
			  	\STATE $Int$ $ b \gets Blue(pixel)$
				\IF{$\neg ( r = 255$ \AND $g = 0$ \AND $b=255)$}
					\STATE $I_{dst}(x+bw,y+bh) = I_{blit}(x+bw,y+bh)$ 
				\ENDIF
			\ENDFOR
		 \ENDFOR
  \end{algorithmic}
  \caption{$blit (I_{src}, I_{dst}, I_{blit})$}
  \label{alg:blit}
\end{algorithm}


\subsubsection{Implementación ASM}

En este filtro procesamos de a 4 píxeles: tenemos dos ciclos, uno que recorre las columnas de la imagen y otra que recorre las filas. Cuando nos encontramos en la fila "Height($I_{src}$) $-$ Height($I_{blit}$)" y la columna "Width($I_{src}$) $-$ Width($I_{blit}$)" es en este momento que tenemos que evaluar si aplicar o no el blit según la fórmula explicada arriba. Abajo detallaremos esta parte del código ASM que desarrolla el blit con un ejemplo.	

\begin{itemize}

	\item En los registros \textbf{xmm0,xmm14} tenemos la copia de los cuatro píxeles que levantamos de memoria, uno es $I_{src}$ y el otro $I_{blit}$ respectivamente.
			Y en \textbf{xmm2, xmm15} las máscaras que utilizamos para la comparación y las operaciones lógicas.

		\begin{center}
		   \begin{tabular}{| c | c | c | c || c | c | c | c || c | c | c | c || c | c | c | c |}
			 \hline
			 a & b & g & r & a & b & g & r & a & b & g & r & a & b & g & r \\ \hline
		   \end{tabular}
		   \\ \textbf{xmm0 $\gets$ $I_{src}$ }
		\end{center}

		\begin{center}
		   \begin{tabular}{| c | c | c | c || c | c | c | c || c | c | c | c || c | c | c | c |}
			 \hline
			 A & B & G & R & A & B & G & R & A & B & G & R & A & B & G & R \\ \hline
		   \end{tabular}
		   \\ \textbf{xmm14 $\gets$ $I_{blit}$ }
		\end{center}
		
		 
		\begin{center}
		   \begin{tabular}{| c | c | c | c || c | c | c | c || c | c | c | c || c | c | c | c |}
			 \hline
			 255 & 255 & 0 & 255 & 255 & 255 & 0 & 255 & 255 & 255 & 0 & 255 & 255 & 255 & 0 & 255 \\ \hline
		   \end{tabular}
		   \\  \textbf{Mascara Magenta xmm15 (maskMagenta)}
		\end{center}

		\begin{center}
		   \begin{tabular}{| c | c | c | c || c | c | c | c || c | c | c | c || c | c | c | c |}
			 \hline
			 00h & 00h & 00h & 00h & 00h & 00h & 00h & 00h & 00h & 00h & 00h & 00h & 00h & 00h & 00h & 00h \\ \hline
		   \end{tabular}
		   \\ \textbf{Mascara en xmm2 (maskCeros)}
		\end{center}
		
	\item Máscara para filtrar píxeles de valor magenta. Suponemos que hay dos píxeles color magenta en $I_{blit}$ a modo de ejemplo en los píxeles 1 y 3 (siguiendo el order de píxel_15,...,píxel_0). A esta máscara la guardamos en xmm12 y xmm15.
	
		\begin{center}
		   \begin{tabular}{| c | c | c | c || c | c | c | c || c | c | c | c || c | c | c | c |}
			 \hline
			 0xFF & 0xFF & 0xFF & 0xFF & 0x00 & 0x00 & 0x00 & 0x00 & 0xFF & 0xFF & 0xFF & 0xFF & 0x00 & 0x00 & 0x00 & 0x00 \\ \hline
		   \end{tabular}
		   \\ \textbf{xmm12,xmm15 $\gets$ pcmpeqd xmm15, xmm14}
		\end{center}

	\item Máscara para filtrar píxeles que no son magenta.

		\begin{center}
		   \begin{tabular}{| c | c | c | c || c | c | c | c || c | c | c | c || c | c | c | c |}
			 \hline
			 0x00 & 0x00 & 0x00 & 0x00 & 0xFF & 0xFF & 0xFF & 0xFF & 0x00 & 0x00 & 0x00 & 0x00 & 0xFF & 0xFF & 0xFF & 0xFF \\ \hline
		   \end{tabular}
		   \\ \textbf{xmm15 $\gets$ pcmpeqd xmm15, xmm13}
		\end{center}

	\item  Me quedo con los valores de $I_{src}$ que tengo que poner en la imagen $I_{dst}$.
		\begin{center}
		   \begin{tabular}{| c | c | c | c || c | c | c | c || c | c | c | c || c | c | c | c |}
			 \hline
			 A & B & G & R & 0x00 & 0x00 & 0x00 & 0x00 & A & B & G & R & 0x00 & 0x00 & 0x00 & 0x00 \\ \hline
		   \end{tabular}
		   \\ \textbf{xmm12 $\gets$ pand xmm12, xmm0}
		\end{center}		

	\item Me quedo con los valores de $I_{blit}$ que tengo que poner en la imagen $I_{dst}$.
		\begin{center}
		   \begin{tabular}{| c | c | c | c || c | c | c | c || c | c | c | c || c | c | c | c |}
			 \hline
			 0x00 & 0x00 & 0x00 & 0x00 & a & b & g & r & 0x00 & 0x00 & 0x00 & 0x00 & a & b & g & r \\ \hline
		   \end{tabular}
		   \\ \textbf{xmm15 $\gets$ pand xmm15, xmm14}
		\end{center}		
	
	\item Junto todos los valores en XMM15.
		\begin{center}
		   \begin{tabular}{| c | c | c | c || c | c | c | c || c | c | c | c || c | c | c | c |}
			 \hline
			 A & B & G & R & a & b & g & r & A & B & G & R & a & b & g & r \\ \hline
		   \end{tabular}
		   \\ \textbf{xmm15 $\gets$ por xmm15, xmm12}
		\end{center}		
	\item Por último resta guardar estos 4 píxeles (XMM15) en $I_{dst}$.

\end{itemize}

Para más detalles dejamos un extracto del código ASM:

\begin{codesnippet}
\begin{verbatim}
	maskMagenta: db 255	 ,0, 255, 255, 255, 0, 255, 255, 255, 0, 255, 255, 255, 0, 255, 255
	maskCero: db 0, 0, 0, 0, 0, 0, 0, 0, 0, 0, 0, 0, 0, 0, 0, 0 

section .text

blit_asm:
		...
		movdqu xmm0, [rdi]; xmm0=|A B G R|A B G R|A B G R|A B G R|; img	
		movdqu xmm14, [r15]; xmm14=|a b g r|a b g r|a b g r|a b g r|; blit
							   ;Filtro los valores color magenta
		pcmpeqd xmm15, xmm14   ; xmm15=|FF FF FF FF|00 00 00 00|FF FF FF FF|00 00 00 00| 
		movdqu xmm12, xmm15	   ; xmm12= xmm15 paso la mascara a xmm12 
							   ;Filtro los valores que no son magenta
		pcmpeqd xmm15, xmm13   ; xmm15=|00 00 00 00|FF FF FF FF|00 00 00 00|FF FF FF FF|
							   ;Me quedo con los velores de xmm0 que tengo que poner en la imagen
		pand xmm12, xmm0       ; xmm2=|A B G R|00 00 00 00|A B G R|00 00 00 00|
							   ;Me quedo con los valores de blit que tengo que poner en la imagen
		pand xmm15, xmm14      ; xmm15=|00 00 00 00|a b g r|00 00 00 00|a b g r|
							   ; Junto los dos valores en xmm15
		por xmm15, xmm12 	   ;xmm15=|A B G R|a b g r|A B G R|a b g r|
		movdqu [rsi], xmm15	   ; Copio todo a dst
		movdqu xmm15, [maskMagenta]
		movdqu xmm13, [maskCero]
		...
\end{verbatim}
\end{codesnippet}
\subsection{Monocromatizar}

\subsubsection{Descripción}

\begin{wrapfigure}{r}{0.3\textwidth}
	\centering
	\includegraphics[width=0.3\textwidth]{imagenes/lenaMONO.jpg}
\end{wrapfigure}

El objetivo de este filtro es convertir una imagen color a una escala de grises. A diferencia de otros filtros, los pixeles de la imagen de salida estarán compuestos por un solo byte, que representará la intensidad de la luz. El criterio para obtener este valor es tomar el máximo entre los componentes R, G, B originales de cada pixel y aplicarlo al correspondiente en la imagen destino. Para esto aplicamos, a todos los pixeles de la imagen original, la siguiente función:

\begin{center}
	$I_{out}(p) = max(R, G, B)$
\end{center}

\hfill

\subsubsection{Implementación C}

A continuación el pseudocódigo de la implementación en C:

\begin{algorithm}[H]
  \begin{algorithmic}[1]
		\FORALL{y:=0 \TO  Height($I_{src}$)}
		 %\FOR x:=0 \TO  Width($I_{src}$)\STEP 1 \DO
			\FORALL{x:=0 \TO  Width($I_{src}$)}
			  \STATE $pixel \gets I_{src}(x,y)$
			  \STATE $Int$ $ r \gets Red(pixel) $
			  \STATE $Int$ $g \gets Green(pixel)$
			  \STATE $Int$ $ b \gets Blue(pixel)$
			  \STATE $Int$ $max \gets max(r, max(g, b))$
			  \STATE $pixel \gets DevolverPixel(r,g,b)$
			  \STATE $I_{dst}(x,y) \gets pixel$
			\ENDFOR
		 \ENDFOR
  \end{algorithmic}
  \caption{$monocromatizar (I_{src}, I_{dst})$}
  \label{alg:monocromatizar}
\end{algorithm}

\subsubsection{Implementación ASM}

En este filtro procesamos de a 4 píxeles en cada iteración usando las instrucciones \textbf{SSE} de la arquitectura. Por cada iteración se realizaron los cálculos del máximo de cada píxel y este se guardó en la imagen de salida.

\begin{itemize}

	\item En los registros \textbf{xmm0,xnm4, xmm5} tenemos la copia de los cuatro píxeles que levantamos de memoria. Y en \textbf{xnm1, xmm2, xmm3} las máscaras que usamos para los shuffles que luego utilizaremos para permutar los componentes.

		\begin{center}
		   \begin{tabular}{| c | c | c | c || c | c | c | c || c | c | c | c || c | c | c | c |}
			 \hline
			 a & b & g & r & a & b & g & r & a & b & g & r & a & b & g & r \\ \hline

		   \end{tabular}
		   \\ \textbf{xmm0, xmm4, xmm5}
		\end{center}
		 
		\begin{center}
		   \begin{tabular}{| c | c | c | c || c | c | c | c || c | c | c | c || c | c | c | c |}
			 \hline
			 15 & 13 & 13 & 13 & 11 & 9 & 9 & 9 & 7 & 5 & 5 & 5 & 3 & 1 & 1 & 1 \\ \hline
		   \end{tabular}
		   \\  \textbf{Mascara en xmm1 (mask1)}
		\end{center}

		\begin{center}
		   \begin{tabular}{| c | c | c | c || c | c | c | c || c | c | c | c || c | c | c | c |}
			 \hline
			 15 & 14 & 14 & 14 & 11 & 10 & 10 & 10 & 7 & 6 & 6 & 6 & 3 & 2 & 2 & 2 \\ \hline
		   \end{tabular}
		   \\ \textbf{Mascara en xmm2 (mask2)}
		\end{center}

		\begin{center}
		   \begin{tabular}{| c | c | c | c || c | c | c | c || c | c | c | c || c | c | c | c |}
			 \hline
			 15 & 12 & 12 & 12 & 11 & 8 & 8 & 8 & 7 & 4 & 4 & 4 & 3 & 0 & 0 & 0 \\ \hline
		   \end{tabular}
		   \\ \textbf{Mascara en xmm3(mask3)}
		\end{center}

	\item Realizamos el \textbf{shuffle xmm4, xmm1} que nos coloca el componente \textbf{g} en las posiciones que podemos observar en el registro xmm4. Y luego calculamos el máximo con la instrucción \textbf{pmaxub}. 

		\begin{center}
		   \begin{tabular}{| c | c | c | c || c | c | c | c || c | c | c | c || c | c | c | c |}
			 \hline
			 a & g & g & g & a & g & g & g & a & g & g & g & a & g & g & g \\ \hline
		   \end{tabular}
		   \\ \textbf{xmm4 $\gets$ pshufb xmm4, xmm1}
		\end{center}


		\begin{center}
		   \begin{tabular}{| c | c | c | c || c | c | c | c || c | c | c | c || c | c | c | c |}
			 \hline
			 a & . & . & max(g,r) & a & . & . & max(g,r) & a & . & . & max(g,r) & a & . & . & max(g,r)  \\ \hline
		   \end{tabular}
		   \\ \textbf{xmm4 $ \gets $ pmaxub xmm4, xmm0}
		\end{center}

	\item Realizamos el mismo procedimiento en este paso. Luego nos quedaría el valor del máximo (ver imagen). 
		\begin{center}
		   \begin{tabular}{| c | c | c | c || c | c | c | c || c | c | c | c || c | c | c | c |}
			 \hline
			 a & b & b & b & a & b & b & b & a & b & b & b & a & b & b & b \\ \hline
		   \end{tabular}
		   \\ \textbf{xmm5 $\gets$ pshufb xmm5, xmm2}
		\end{center}

		\begin{center}
		   \begin{tabular}{| c | c | c | c || c | c | c | c || c | c | c | c || c | c | c | c |}
			 \hline
			 a & . & . & max(g,r,b) & a & . & . & max(g,r,b) & a & . & . & max(g,r,b) & a & . & . & max(g,r,b)  \\ \hline
		   \end{tabular}
		   \\ \textbf{xmm4 $\gets$ pmaxub xmm4, xmm5}
		\end{center}


	\item Solo resta copiar los máximos. Para esto utilizaremos el shuffle junto con la máscara contenida en xmm3.

		\begin{center}
		   \begin{tabular}{| c | c | c | c || c | c | c | c || c | c | c | c || c | c | c | c |}
			 \hline
			 a & max & max & max & a & max & max & max & a & max & max & max & a & max & max & max  \\ \hline
		   \end{tabular}
		   \\ \textbf{xmm5 $\gets$ pshufb xmm4, xmm3}
		\end{center}

\end{itemize}

Por último resta copiar esto a memoria, brindamos el código ASM de este paso.

\begin{codesnippet}
\begin{verbatim}
section .data
	mask1: db 1, 1, 1, 3, 5, 5, 5, 7, 9, 9, 9, 11, 13, 13, 13, 15
	mask2: db 2, 2, 2, 3, 6, 6, 6, 7, 10, 10, 10, 11, 14, 14, 14, 15
	mask3: db 0, 0, 0, 3, 4, 4, 4, 7, 8, 8, 8, 11, 12, 12, 12, 15

section .text

monocromatizar_inf_asm:
.........
		movdqu xmm0, [rdi]; xmm0=|a b g r|a b g r|a b g r|a b g r|
		movdqu xmm4, xmm0; xmm4= xmm0
		movdqu xmm5, xmm0; xmm5= xmm0

		pshufb xmm4, xmm1; xmm4=|a g g g|a g g g|a g g g|a g g g|
		pmaxub xmm4, xmm0; xmm4=|a . . max(g,r)|a . . max(g,r)|a . . max(g,r)|a . . max(g,r)|;1er máximo
 
		pshufb xmm5, xmm2; xmm5=|a b b b|a b b b|a b b b|a b b b|			
		pmaxub xmm4, xmm5; xmm4=|a . . max(g,r,b)|a . . max(g,r,b)|a . . max(g,r,b)|;máx de máximos
						 ; xmm4=|a . . max|a . . max|a . . max|a . . max|	
		pshufb xmm4, xmm3; xmm4=|a max max max|a max max max|a max max max|a max max max|

		movdqu [rsi], xmm4; [rsi]= xmm4
......
\end{verbatim}
\end{codesnippet}
\subsection{Ondas}


\begin{algorithm}[H]
  \begin{algorithmic}[1]
    \FORALL{pixel ubicado en la posici'on $\mathbf{(x, y)}$}
      \STATE $d_x \gets x - x_0$

      \STATE

      \STATE $d_y \gets y - y_0$

      \STATE

      \STATE $d_{xy} \gets \sqrt{d_{x}^2+d_{y}^2}$

      \STATE

      \STATE $r \gets \frac{(d_{xy} - RADIUS)}{WAVELENGTH}$

      \STATE

      \STATE $a \gets \frac{1}{1 + (\frac{r}{TRAINWIDTH})^2 }$

      \STATE

      \STATE $t \gets ( r-floor(r) ) \cdot 2 \cdot \pi - \pi$

      \STATE

      \STATE $prof \gets a \cdot (t - \frac{t^3}{6}+\frac{t^5}{120}-\frac{t^7}{5040})$

      \STATE

      \STATE $pixel = prof \cdot 64 + I_{src}(x, y)$    

      \STATE

      \STATE $I_{dst}(x, y) = saturar(pixel)$
    \ENDFOR
  \end{algorithmic}
  \caption{$ondas (I_{src}, I_{dst}, x_0, y_0)$}
  \label{alg:ondas}
\end{algorithm}

donde:

\begin{itemize}
  \item $x_0$ e $y_0$ representan la posici'on donde est'a centrada la onda,
  \item $RADIUS$, $WAVELENGTH$ y $TRAINWIDTH$ son constantes que definen la 
  forma de la onda y
  \item $saturar(x)$ es una funci'on que retorna $0$ si $x$ es menor $0$, $255$
  si es mayor a $255$ y $x$ en cualquier otro caso.
\end{itemize}









\begin{codesnippet}
\begin{verbatim}


#define PI 			3.1415
#define RADIUS 		35
#define WAVELENGTH 	64
#define TRAINWIDTH 	3.4

float sin_taylor (float x) {
    float x_3 = x*x*x;
    float x_5 = x*x*x*x*x;
    float x_7 = x*x*x*x*x*x*x;

    return x-(x_3/6.0)+(x_5/120.0)-(x_7/5040.0);
}

float profundidad (int x, int y, int x0, int y0) {
    float dx = x - x0;
    float dy = y - y0;

    float dxy = sqrt(dx*dx+dy*dy);

    float r = (dxy-RADIUS)/WAVELENGTH ;
    float k = r-floor(r);
    float a = 1.0/(1.0+(r/TRAINWIDTH)*(r/TRAINWIDTH));

    float t = k*2*PI-PI;

    float s_taylor = sin_taylor(t);

    return a * s_taylor;
}

float saturar(float color) {
    if (color < 0) {
        color = 0;
    } else if (color > 255) {
        color = 255;
    }
    return color;
}

void ondas_c (
	unsigned char *src,
	unsigned char *dst,
	int width,
	int height,
	int src_row_size,
	int dst_row_size,
	int x0,
	int y0
) {
    unsigned char (*src_matrix)[src_row_size] = (unsigned char (*)[src_row_size]) src;
    unsigned char (*dst_matrix)[dst_row_size] = (unsigned char (*)[dst_row_size]) dst;

    for (int i = 0; i < height; i++) {
        for (int j = 0; j < width * 4; j += 4) {
            float azul = profundidad(i, j / 4, x0, y0) * 64 + src_matrix[i][j];
            azul = saturar(azul);
            dst_matrix[i][j] = azul;

            float verde = profundidad(i, j / 4, x0, y0) * 64 + src_matrix[i][j + 1];
            verde = saturar(verde);
            dst_matrix[i][j + 1] = (unsigned char) verde;

            float rojo = profundidad(i, j / 4, x0, y0) * 64 + src_matrix[i][j + 2];
            rojo = saturar(rojo);
            dst_matrix[i][j + 2] = (unsigned char) rojo;
        }
    }
}
\end{verbatim}
\end{codesnippet}

\subsection{Temperature}

\subsubsection{Descripción}

\begin{wrapfigure}{r}{0.3\textwidth}
	\centering
	\includegraphics[width=0.3\textwidth]{imagenes/lenaTEMP.jpg}
\end{wrapfigure}

El filtro temperatura toma una imagen fuente y genera un efecto que simula un mapa de calor. Dicho efecto lo consigue tomando los tres componentes de color de cada pixel y promediándolos. Luego le asigna un valor que depende de este promedio \textit{t}.

\begin{center}
$\mathsf{t}_{(i,j)} = \lfloor(\mathsf{src}.r_{(i,j)} + \mathsf{src}.g_{(i,j)} + \mathsf{src}.b_{(i,j)}) / 3\rfloor$
\end{center}

Finalmente el color en la imagen destino se determina en función de la temperatura \textit{t} conforme al siguiente mapeo:

\begin{center}
\begin{displaymath}
\mathsf{dst}_{(i,j)}<r,g,b> = \left\{
\begin{array}{l l}
			<0,0, 128 + t \cdot 4> & \text{si }t < 32\\
			<0, (t - 32) \cdot 4, 255> & \text{si }32 \le t < 96\\
			<(t-96) \cdot 4, 255, 255 - (t-96) \cdot 4> & \text{si }96 \le t < 160\\
			<255, 255 - (t - 160) \cdot 4, 0> & \text{si }160 \le t < 224\\
			<255 - (t - 224) \cdot 4, 0 , 0> & \text{si no} \\
\end{array}
\right.
\end{displaymath}
\end{center}

\hfill
\subsubsection{Implementación C}

El algoritmo implementado en lenguaje C recorre la imagen iterativamente. Por cada pixel en la imagen original calcula el promedio de sus componentes RGB (línea 4) y según este valor se guarda en la imagen destino el calculo correspondiente a la temperatura (líneas 5 a 15). A continuación el pseudocódigo:
	
\begin{algorithm}[H]
  \begin{algorithmic}[1]
		\FORALL{y:=0 \TO  Height($I_{src}$)}
			\FORALL{x:=0 \TO  Width($I_{src}$)}
				\STATE $ pixel \gets I_{src}(x,y)$
				\STATE $Nat $ $ t \gets \lfloor(\frac{Red(pixel)+Green(pixel)+Blue(pixel)}{3}\rfloor$
				\IF{$t < 32$}
					\STATE $I_{dst}(x,y) \gets DevolverPixel(0,0,128+t \cdot 4)$
				\ELSIF{$ 32 \leq t < 96$}
					\STATE $I_{dst}(x,y) \gets DevolverPixel(0,(t-32) \cdot 4,255)$
				\ELSIF{$ 96 \leq t < 160$}
					\STATE $I_{dst}(x,y) \gets DevolverPixel((t-96) \cdot 4,255, 255-(t-96) \cdot 4)$
				\ELSIF{$ 160 \leq t < 224$}
					\STATE $I_{dst}(x,y) \gets DevolverPixel(255, 255-(t-160) \cdot 4, 0)$
				\ELSE		
					\STATE $I_{dst}(x,y) \gets DevolverPixel(255-(t-224) \cdot 4, 0, 0)$
				\ENDIF	
			\ENDFOR
		 \ENDFOR
  \end{algorithmic}
  \caption{$temperature (I_{src}, I_{dst})$}
  \label{alg:temperature}
\end{algorithm}	

\subsubsection{Implementación ASM}
En este filtro procesaremos de a 4 píxeles. Daremos la explicación de la partes más significativas del código assembler.

\subsubsection*{Cálculo del valor T}

En este tramo de código traigo los 4 píxeles de memoria. Desempaquetamps de Byte a Word en los registros XMM2(parte baja) y XMM1 (parte ata). Después shifteamos 16 bits a izquierda y luego a derecha cada paquete de QWords para de esta manera tener el valor de la trasparencia en 0.

\begin{codesnippet}
\begin{verbatim}
    movdqu xmm0, [rdi]      ; xmm0 = |a3 b3 g3 r3|a2 b2 g2 r2|a1 b1 g1 r1|a0 b0 g0 r0| Píxeles originales
    movdqu xmm1, xmm0       ; xmm1= xmm0
    movdqu xmm2, xmm0       ; xmm2= xmm0

    pxor xmm7, xmm7         ; Masacara de ceros en xmm7 para desempaquetar de byte a word
    punpcklbw xmm2, xmm7    ; xmm2 = |a1 b1 g1 r1|a0 b0 g0 r0| parte baja
    punpckhbw xmm1, xmm7    ; xmm1 = |a3 b3 g3 r3|a2 b2 g2 r2| parte alta
                            ; Shifteo a izquierda para olvidarme del alfa
    psllq xmm1, 16          ; xmm1=|b3 g3 r3 0|b2 g2 r2 0|
    psllq xmm2, 16          ; xmm2=|b1 g1 r1 0|b0 g0 r0 0|
    psrlq xmm1, 16          ; xmm1=|0 b3 g3 r3|0 b2 g2 r2|
    psrlq xmm2, 16          ; xmm2=|0 b1 g1 r1|0 b0 g0 r0|
\end{verbatim}
\end{codesnippet}

Para calcular las sumas de la 3 componentes \emph{b+g+r}. Realizamos sumas horizontales con la intrucción \emph{phaddw} y colocamos las sumas en los registros XMM1 y XMM2.

\begin{codesnippet}
\begin{verbatim}
                       ; Sumas horizontal
    phaddw xmm2, xmm2  ; xmm2=|0+b1 g1+r1 0+b0 g0+r0|0+b1 g1+r1 0+b0 g0+r0|
    phaddw xmm1, xmm1  ; xmm1=|0+b3 g3+r3 0+b2 g2+r2| 0+b3 g3+r3 0+b2 g2+r2|

    phaddw xmm2, xmm2  ; xmm2 =|b1+g1+r1 b0+g0+r0 b1+g1+r1 b0+g0+r0|b1+g1+r1 b0+g0+r0 b1+g1+r1 b0+g0+r0|
    phaddw xmm1, xmm1  ; xmm1 =|b3+g3+r3 b2+g2+r2 b3+g3+r3 b2+g2+r2|b3+g3+r3 b2+g2+r2 b3+g3+r3 b2+g2+r2|
                       ; xmm2 =|s1 s0 s1 s0|s1 s0 s1 s0|
                       ; xmm1 =|s3 s2 s3 s2|s3 s2 s3 s2|
\end{verbatim}
\end{codesnippet}

El objetivo de este paso es obtener \emph{xmm1=$|s3|s2|s1|s0|$}, donde \emph{s3,s2,s1 y s0} son las sumas \emph{b+g+r} del píxel 3,..,píxel 1 respectivamente.

\begin{codesnippet}
\begin{verbatim}
                                  ;shuffle parte baja (bits 63 a 0) de los words, para reordenar las sumas s0,s1,s2ys3
                                  ; imm8 = 1 1 0 0
    pshufhw xmm2, xmm2, 01010000b ; xmm2=|s1 s0 s1 s0|s1 s1 s0 s0| 
    pshufhw xmm1, xmm1, 01010000b ; xmm1=|s3 s2 s3 s2|s3 s3 s2 s2|
                                  ;shuffle parte alta(bits 127 a 64) de los word
    pshuflw xmm2, xmm2, 01010000b ; xmm2=|s1 s1 s0 s0|s1 s1 s0 s0|
    pshuflw xmm1, xmm1, 01010000b ; xmm1=|s3 s3 s2 s2|s3 s3 s2 s2|
                                  ; observar q las sumas s0,...,s1 miden words, shifteo a derecha c/paquete dobleWord 16 bits
    psrld xmm2, 16;               ; xmm2=|0 s1|0 s0|0 s1|0 s0| 
    psrld xmm1, 16                ; xmm1=|0 s3|0 s2|0 s3|0 s2| 
								  ; Que es lo mismo q decir
                                  ; xmm2=|s1|s0|s1|s0|
                                  ; xmm1=|s3|s2|s3|s2|
\end{verbatim}
\end{codesnippet}

El objetivo de este paso es obtener \emph{xmm1=$|s3|s2|s1|s0|$}. Para eso primero convertimos los paquetes de enteros de Doble Word a Floats usando la intrucción \emph{cvtdq2ps}. Luego usamos la funcion \emph{blenps}, que sive para mezclar los registos XMM1 y XMM2 según el valor que pongamos en la fuente número 2(en este caso 0011b).

\begin{codesnippet}
\begin{verbatim}
                                ; Convierto los valores de xmm1 a float
    cvtdq2ps xmm1, xmm1         ; xmm1=|s3|s2|s3|s2|
    cvtdq2ps xmm2, xmm2         ; xmm2=|s1|s0|s1|s0|
                                ; mesclamos los registros xmm1 y xmm2 usando el blendps
    blendps xmm1, xmm2, 0011b   ; xmm1 =|s3|s2|s1|s0|
\end{verbatim}
\end{codesnippet}

En este último paso queremos obtener \emph{xmm1=$|t4|t3|t2|t1|$}. Para eso dividimos (cvtps2dq) cada paquete de floats por 3. Convertimos los Floats a enteros dobleWords (cvtps2dq). Después desempaquetamos de Doble Word a Words (packusdw) y de Word a Byte (packuswb). Como último paso usamos shuffle (pshufb), para permutar los paquetes de Bytes teniendo como máscara xmm10. 

\begin{codesnippet}
\begin{verbatim}
    ; <t, t, t, t>          t < 32   <0, 0, 0, 4t + 128>
                            ; Divido todo por 3
    divps xmm1, xmm8        ; xmm1 =|s3/3.0|s2/3.0|s1/3.0|s0/3.0| 
                            ; convierto de Float a dobleWord entero
    cvtps2dq xmm1, xmm1     ; xmm1 =|t3|t2|t1|t0|
                            ; desempaqueto de dobleWord a Word sin signo y con saturación
    packusdw xmm1, xmm1     ; xmm1 =|t3|t2|t1|t0|t3|t2|t1|t0|
                            ; desempaqueto Word sin signo y byte con saturación
    packuswb xmm1, xmm1     ; xmm1 =|t3|t2|t1|t0|t3|t2|t1|t0|t3|t2|t1|t0|t3|t2|t1|t0| 
                            ; xmm10=|3|3|3|3|2|2|2|2|1|1|1|1|0|0|0|0|
    pshufb xmm1, xmm10      ; xmm1 = |t3t3|t3|t3|t2|t2|t2|t2|t1|t1|t1|t1|t0|t0|t0|t0|
                            ; xmm1 = |t3 t3 t3 t3|t2 t2 t2 t2|t1 t1 t1 t1|t0 t0 t0 t0|
\end{verbatim}
\end{codesnippet}

\subsubsection*{Caso $t<32$}
El objetivo de este caso es obtner en \emph{xmm4=$|$0 0 0 4.t3+128$|$0 0 0 4.t2+128$|$0 0 0 4.t1+128$|$0 0 0 4.t0+128$|$}. Primero multiplicamos por 4. Luego sumamos 128 de manera tal me quede $4.t+128$. Shifteamos cada paquete de DobleWord a derecha 24 bits. Finalmente guardamos todo en el registro xmm4 (ver fragmento de código).

\begin{codesnippet}
\begin{verbatim}
                        ; Caso: t < 32    <0, 0, 0, 4t + 128>
    movdqu xmm4, xmm1   ; xmm4=|t3 t3 t3 t3|t2 t2 t2 t2|t1 t1 t1 t1|t0 t0 t0 t0|
                        ; multiplico por 2
    paddb  xmm4, xmm1   ; xmm4=|2.t3 2.t3 2.t3 2.t3|2.t2 2.t2 2.t2 2.t2|2.t1 2.t1 2.t1 2.t1|2.t0 2.t0 2.t0 2.t0|   
    paddb  xmm4, xmm1   ; multiplico por 3 y 4
    paddb  xmm4, xmm1   ; xmm4=|4.t3 4.t3 4.t3 4.t3|4.t2 4.t2 4.t2 4.t2|4.t1 4.t1 4.t1 4.t1|4.t0 4.t0 4.t0 4.t0| 
                        ; sumo 128 a cada byte de xmm4
                        ; xmm1=|128|128|128|128|128|128|128|128|128|128|128|128|128|128|128|128|
    paddb  xmm4, xmm11  ; xmm4=|4.t3+128 4.t3+128 4.t3+128 4.t3+128|4.t2+128 4.t2+128 4.t2+128 4.t2+128
                        ;      |4.t1+128 4.t1+128 4.t1+128 4.t1+128|4.t0+128 4.t0+128 4.t0+128 4.t0+128|
                        ; shifteo a derecha 24 bits a cada paquete de DobleWord
    psrld  xmm4, 24     ; xmm4=|0 0 0 4.t3+128|0 0 0 4.t2+128|0 0 0 4.t1+128|0 0 0 4.t0+128|
\end{verbatim}
\end{codesnippet}

\subsubsection*{Casos restantes}
Caso: $32 \leq t < 96$ \\
En este caso debemos obtener \emph{xmm5=$|$0 0 4.(t3-32) 255$|$...$|$0 0 4.(t0-32) 255$|$ }. 
La parte interezante es que usamos la instrucción \emph{pinserb}, que inserta el valor contenido en \emph{r9}.
\begin{codesnippet}
\begin{verbatim}				
                        ;xmm11=|128|128|128|128|128|128|128|128|128|128|128|128|128|128|128|128|
                        ;xmm4=|t3 t3 t3 t3|t2 t2 t2 t2|t1 t1 t1 t1|t0 t0 t0 t0|
    psubb xmm5, xmm11   ;xmm5=|4.t3-128 4.t3-128 4.t3-128 4.t3-128|4.t2-128 4.t2-128 4.t2-128 4.t2-128
                        ;     |4.t1-128 4.t1-128 4.t1-128 4.t1-128|4.t0-128 4.t0-128 4.t0-128 4.t0-128|
                        ;xmm5=|4.(t3-32) 4.(t3-32) 4.(t3-32) 4.(t3-32)|....|4.t0-32 4.4.t0-32 4.4.t0-32 4.4.t0-32|                           
                        ; r9b=255
    pinsrb xmm5, r9b, 12 ; xmm5=|4.(t3-32) 4.(t3-32) 4.(t3-32) 4.(t3-32)+255|...|4.t0-32 4.4.t0-32 4.4.t0-32 4.4.t0-32|
    pinsrb xmm5, r9b, 8  
    pinsrb xmm5, r9b, 4
    pinsrb xmm5, r9b, 0  ;xmm5=|4.(t3-32) 4.(t3-32) 4.(t3-32) 255|...|4.(t0-32) 4.(t0-32) 4.(t0-32) 255|                    

    pslld xmm5, 16      ;xmm5=|4.(t3-32) 255 0 0|...|4.(t0-32) 255 0 0|
    psrld xmm5, 16      ;xmm5=|0 0 4.(t3-32) 255|...|0 0 4.(t0-32) 255|                    
\end{verbatim}
\end{codesnippet}

Casos restantes se resuelven de igual manera($96\leq t < 160$, $160\leq t <225$ y  $224 \leq t$).

\subsubsection*{Paso final}

Este es el último paso. Acá debemos quedarnos con las valores según valfa \emph{T}, Son todas operaciones lógicas y comparaciones. Solo debemos recordar q en xmm4,  xmm5, xmm6, xmm7 y xmm3 estan los casos T.

\begin{codesnippet}
\begin{verbatim}				
    psrld xmm1,  24    ; xmm1=|0 0 0 t3|0 0 0 0 t2|0 0 0 t1|0 0 0 t0|
							; En este paso produzco todas las máscaras segun el valor de T
    pcmpgtd xmm12, xmm1     ; [32 > t?...] Me quedo con t < 32
    pcmpgtd xmm13, xmm1     ; [96 > t?...] Me quedo con t < 96
    pxor    xmm13, xmm12    ; Me quedo con 32 <= t < 96
    pcmpgtd xmm14, xmm1     ; [160 > t?...] Me quedo con t < 160
    pxor    xmm14, xmm12
    pxor    xmm14, xmm13    ; Me quedo con 96 <= t < 160
    pcmpgtd xmm15, xmm1     ; [224 > t?...] Me quedo con t < 224
    pxor    xmm15, xmm12
    pxor    xmm15, xmm13
    pxor    xmm15, xmm14    ; Me quedo con 160 <= t < 224    ; t > 224?
    pxor    xmm9,  xmm12
    pxor    xmm9,  xmm13
    pxor    xmm9,  xmm14
    pxor    xmm9,  xmm15
							; utilizando las mascaras producidas en el paso anterior 
							; unimos todos los casos
    pand xmm12, xmm4        ; [t<32, 0, 0 , t<32, ...]
    pand xmm13, xmm5        ; [t<96, t <96, 0,...]
    pand xmm14, xmm3        ; [t<160, t<160, 0, 0 ,t<160]
    pand xmm15, xmm6        ; [t<224, t<224, 0, 0 ,t<224]
    pand  xmm9, xmm7        ; [t>224, t>224, 0, 0 ,t>224]
													
    por xmm12, xmm13
    por xmm12, xmm14
    por xmm12, xmm15
    por xmm12, xmm9        ;En xmm12 guardamos las 4 temperaturas del pixel.
\end{verbatim}
\end{codesnippet}

Resta copiar los valores de los 4 píxeles de \emph{xmm12} a memoria.
\subsection{Edge}

% Puede definirse como un borde a los p\'ixeles donde la intensidad de la imagen cambia de forma abrupta. 
% Si se considera una funci\'on de intensidad de la imagen, entonces lo que se busca son saltos en dicha funci\'on.
% La idea b\'asica detr\'as de cualquier detector de bordes es el c\'alculo de un operador local de derivaci\'on. \\
 

% Vamos a usar en este caso el operador de Laplace, cuya matriz es

% $$ M = \left(
% \begin{matrix}
%     0.5 & 1 & 0.5 \\
%     1 & -6 & 1 \\
%     0.5 & 1 & 0.5
% \end{matrix}
% \right)$$


% Para obtener los bordes, se posiciona el centro de la matr\'iz de 
% Laplace en cada posici\'on $(x, y)$ y se realiza la siguiente operaci\'on

% $$dst(x, y) = \sum_{k = 0}^2 \sum_{l = 0}^2 src(x + k - 1, y + l - 1) * M(k, l)$$

% Es decir $dst(x, y) = $
% \begin{center}
%     $\begin{matrix}
%         src(x - 1, y - 1) * M_{00} & + & src(x - 1, y) * M_{01} & + & src(x - 1, y + 1) * M_{02} & +\\
%         src(x, y - 1) * M_{10} & + & src(x, y1) * M_{11} & + & src(x, y + 1) * M_{12} & +\\ 
%         src(x + 1, y - 1) * M_{20} & + & src(x + 1, y) * M_{21} & + & src(x + 1, y + 1) * M_{22} &
%     \end{matrix}$
% \end{center}

Este filtro opera sobre imágenes en escala de grises (1 componente de color por píxel).
En los bordes que no se pueden procesar según la fórmula anterior (por estar a los costados de la
imagen), aplica la fórmula $dst(x, y) = src(x,y)$.


\begin{algorithm}[H]
  \begin{algorithmic}[1]
  
		\FORALL{y:=1 \TO  Height($I_{src}$$-1$)}
		
			\FORALL{x:=1 \TO  Width($I_{src}$)$-1$}
			  
			  \STATE $m_{0,0} \gets I_{src}(x-1,y-1)*0.5$
			  \STATE $m_{0,1} \gets I_{src}(x-1,y)*1$
			  \STATE $m_{0,2} \gets I_{src}(x-1,y+1)*0.5$
			  \STATE $m_{1,0} \gets I_{src}(x,y-1)*1$
			  \STATE $m_{1,1} \gets I_{src}(x,y)*(-6)$
			  \STATE $m_{1,2} \gets I_{src}(x,y+1)*1$
			  \STATE $m_{2,0} \gets I_{src}(x+1,y-1)*0.5$
			  \STATE $m_{2,1} \gets I_{src}(x+1,y)*1$
			  \STATE $m_{2,2} \gets I_{src}(x+1,y+1)*0.5$
			  \STATE $edge \gets m_{0,0}+m_{0,1}+m_{0,2}+m_{1,0}+m_{1,1}+m_{1,2}+m_{2,0}+m_{2,1}+m_{2,2}$
		
			  \STATE $I_{dst}(x,y) \gets Saturar(edge)$
			  
			\ENDFOR

		 \ENDFOR

  \end{algorithmic}
  \caption{$edge (I_{src}, I_{dst})$}
  \label{alg:edge}
\end{algorithm}




\section{Mediciones y rendimiento}

La forma de medir el rendimiento de nuestras implementaciones se realizará por medio de la toma de tiempos de ejecución del algoritmo (sea este el codigo version asembler o el codigo C). Como los tiempos de ejecución son muy pequeños, se utilizará uno de los contadores de performance que posee el procesador.
La instrucción de assembler rdtsc permite obtener el valor del Time Stamp Counter (TSC) del procesador. Este registro se incrementa en uno con cada ciclo del procesador. Obteniendo la diferencia entre los contadores antes y después de la llamada a la función, podemos obtener la cantidad de ciclos de esa ejecución. Esta cantidad de ciclos no es siempre igual entre invocaciones de la función, ya que este registro es global del procesador y se ve afectado por una serie de factores. \newline

Existen principalmente dos problemáticas a solucionar:
1). La ejecución puede ser interrumpida por el scheduler para realizar un cambio de contexto,
esto implicará contar muchos más ciclos (outliers) que si nuestra función se ejecutara sin
interrupciones.

2). Los procesadores modernos varían su frecuencia de reloj, por lo que la forma de medir
ciclos cambiará dependiendo del estado del procesador.
\newline

\textbf{solucion 1):} Para evitar el problema de los ciclos outliers lo que hicimos fue,

\begin{itemize}
	\item[Paso 1:] En nuestro caso hicimos 100 veces la medición de tiempo de nuestro algoritmo y guardarlo en un contenedor(podría ser un arreglo,lista, conjunto, diccionario,.., etc).
	\item[Paso 2:] Sacar la media, tambien conocido como promedio, ejemplito:
		\begin{center} $ Prom =\frac{x_1+x_2+...+x_{100}}{100}$ \end{center}
		Donde $x_i$ es la medición de tiempo de la medición número $i$, con $1 \leq i \leq 100$ 
	\item[Paso 3:] Calculamos la varianza: 			
				\begin{center}
					$Varianza = \sigma^2 = \frac{(x_1 - Prom)+ (x_2 - prom)+ ...+ (x_{100} - prom)}{100} $
				\end{center}
	\item[Paso 4:] Calculamos el desvio estandar,  $\sigma = \sqrt{Varianza}$
	\item[Paso 5:] Utilizando el desvio estandar y el promedio, puedo ver que medición es "buena"  \\ y cual no. Más formalmente una medicion es "buena" si cumple: 
					\begin{center}
					$Prom - \sigma \leq x_i \leq Prom + \sigma $. %%\newline
					\end{center}
	 Luego sumando las todas las mediciones  "buenas" \\ y dividiendalas por la cantidad de mediciones buenas, obtengo el "promedio bueno". Con esto amortiguaria la cantidad de outliers de mis mediciones. 			
\end{itemize}

\textbf{Observar:} Todo lo anterior sirve también para mas de 100 mediciones. \newline

\textbf{Solucion 2):} La solución que planteamos para esto fue, ejecutar sólamente el algoritmo. Con esto queremos decir que la ejecución estara en el nivel más alto de privilegio de ejecución. Esto lo hacemos metiendonos en el sistema operativo (en este caso ubuntu 14.04), tocando el monitor de sistema para darle privilegio a la ejecución. Tambien evitamos interumpir la maquina de forma mecanica(osea humana).  




\section{Contexto}

\paragraph{\textbf{Titulo del parrafo} } Bla bla bla bla.
Esto se muestra en la figura~\ref{nombreparareferenciar}.



\begin{codesnippet}
\begin{verbatim}

struct Pepe {

    ...

};

\end{verbatim}
\end{codesnippet}


\section{Enunciado y solucion} 
%\input{enunciado}
%\input{1-intro}

\section{Conclusiones y trabajo futuro}


\end{document}

