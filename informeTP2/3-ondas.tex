\subsection{Ondas}

Este filtro combina la imagen original con una imagen de ondas, dando tonos más oscuros y mas claros en forma de onda. Estas se generan desde el centro de la imagen hacia sus bordes de manera concéntrica.

Para ellos aplicamos la profundidad y lo sumamos a cada componete del pixel(ver linea 16 y 18 del pseudocodigo). 

\begin{algorithm}[H]
  \begin{algorithmic}[1]
    \FORALL{pixel ubicado en la posici'on $\mathbf{(x, y)}$}
      \STATE $d_x \gets x - x_0$

      \STATE

      \STATE $d_y \gets y - y_0$

      \STATE

      \STATE $d_{xy} \gets \sqrt{d_{x}^2+d_{y}^2}$

      \STATE

      \STATE $r \gets \frac{(d_{xy} - RADIUS)}{WAVELENGTH}$

      \STATE

      \STATE $a \gets \frac{1}{1 + (\frac{r}{TRAINWIDTH})^2 }$

      \STATE

      \STATE $t \gets ( r-floor(r) ) \cdot 2 \cdot \pi - \pi$

      \STATE

      \STATE $prof \gets a \cdot (t - \frac{t^3}{6}+\frac{t^5}{120}-\frac{t^7}{5040})$

      \STATE

      \STATE $pixel = prof \cdot 64 + I_{src}(x, y)$    

      \STATE

      \STATE $I_{dst}(x, y) = saturar(pixel)$
    \ENDFOR
  \end{algorithmic}
  \caption{$ondas (I_{src}, I_{dst}, x_0, y_0)$}
  \label{alg:ondas}
\end{algorithm}

donde:

\begin{itemize}
  \item $x_0$ e $y_0$ representan la posici'on donde est'a centrada la onda,
  \item $RADIUS$, $WAVELENGTH$ y $TRAINWIDTH$ son constantes que definen la 
  forma de la onda y
  \item $saturar(x)$ es una funci'on que retorna $0$ si $x$ es menor $0$, $255$
  si es mayor a $255$ y $x$ en cualquier otro caso.
\end{itemize}

donde:

\begin{itemize}
  \item $x_0$ e $y_0$ representan la posici'on donde est'a centrada la onda,
  \item $RADIUS$, $WAVELENGTH$ y $TRAINWIDTH$ son constantes que definen la 
  forma de la onda y
  \item $saturar(x)$ es una funci'on que retorna $0$ si $x$ es menor $0$, $255$
  si es mayor a $255$ y $x$ en cualquier otro caso.
\end{itemize}
