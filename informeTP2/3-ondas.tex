\subsection{Ondas}


\begin{algorithm}[H]
  \begin{algorithmic}[1]
    \FORALL{pixel ubicado en la posici'on $\mathbf{(x, y)}$}
      \STATE $d_x \gets x - x_0$

      \STATE

      \STATE $d_y \gets y - y_0$

      \STATE

      \STATE $d_{xy} \gets \sqrt{d_{x}^2+d_{y}^2}$

      \STATE

      \STATE $r \gets \frac{(d_{xy} - RADIUS)}{WAVELENGTH}$

      \STATE

      \STATE $a \gets \frac{1}{1 + (\frac{r}{TRAINWIDTH})^2 }$

      \STATE

      \STATE $t \gets ( r-floor(r) ) \cdot 2 \cdot \pi - \pi$

      \STATE

      \STATE $prof \gets a \cdot (t - \frac{t^3}{6}+\frac{t^5}{120}-\frac{t^7}{5040})$

      \STATE

      \STATE $pixel = prof \cdot 64 + I_{src}(x, y)$    

      \STATE

      \STATE $I_{dst}(x, y) = saturar(pixel)$
    \ENDFOR
  \end{algorithmic}
  \caption{$ondas (I_{src}, I_{dst}, x_0, y_0)$}
  \label{alg:ondas}
\end{algorithm}

donde:

\begin{itemize}
  \item $x_0$ e $y_0$ representan la posici'on donde est'a centrada la onda,
  \item $RADIUS$, $WAVELENGTH$ y $TRAINWIDTH$ son constantes que definen la 
  forma de la onda y
  \item $saturar(x)$ es una funci'on que retorna $0$ si $x$ es menor $0$, $255$
  si es mayor a $255$ y $x$ en cualquier otro caso.
\end{itemize}









\begin{codesnippet}
\begin{verbatim}


#define PI 			3.1415
#define RADIUS 		35
#define WAVELENGTH 	64
#define TRAINWIDTH 	3.4

float sin_taylor (float x) {
    float x_3 = x*x*x;
    float x_5 = x*x*x*x*x;
    float x_7 = x*x*x*x*x*x*x;

    return x-(x_3/6.0)+(x_5/120.0)-(x_7/5040.0);
}

float profundidad (int x, int y, int x0, int y0) {
    float dx = x - x0;
    float dy = y - y0;

    float dxy = sqrt(dx*dx+dy*dy);

    float r = (dxy-RADIUS)/WAVELENGTH ;
    float k = r-floor(r);
    float a = 1.0/(1.0+(r/TRAINWIDTH)*(r/TRAINWIDTH));

    float t = k*2*PI-PI;

    float s_taylor = sin_taylor(t);

    return a * s_taylor;
}

float saturar(float color) {
    if (color < 0) {
        color = 0;
    } else if (color > 255) {
        color = 255;
    }
    return color;
}

void ondas_c (
	unsigned char *src,
	unsigned char *dst,
	int width,
	int height,
	int src_row_size,
	int dst_row_size,
	int x0,
	int y0
) {
    unsigned char (*src_matrix)[src_row_size] = (unsigned char (*)[src_row_size]) src;
    unsigned char (*dst_matrix)[dst_row_size] = (unsigned char (*)[dst_row_size]) dst;

    for (int i = 0; i < height; i++) {
        for (int j = 0; j < width * 4; j += 4) {
            float azul = profundidad(i, j / 4, x0, y0) * 64 + src_matrix[i][j];
            azul = saturar(azul);
            dst_matrix[i][j] = azul;

            float verde = profundidad(i, j / 4, x0, y0) * 64 + src_matrix[i][j + 1];
            verde = saturar(verde);
            dst_matrix[i][j + 1] = (unsigned char) verde;

            float rojo = profundidad(i, j / 4, x0, y0) * 64 + src_matrix[i][j + 2];
            rojo = saturar(rojo);
            dst_matrix[i][j + 2] = (unsigned char) rojo;
        }
    }
}
\end{verbatim}
\end{codesnippet}
